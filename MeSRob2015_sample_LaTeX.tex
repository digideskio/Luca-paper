%%%%%%%%%%%%%%%%%%%% author.tex %%%%%%%%%%%%%%%%%%%%%%%%%%%%%%%%%%%
%
% sample root file for your "contribution" to a contributed volume
%
% Use this file as a template for your own input.
%
%%%%%%%%%%%%%%%%%%%% author.tex %%%%%%%%%%%%%%%%%%%%%%%%%%%%%%%%%%


% RECOMMENDED %%%%%%%%%%%%%%%%%%%%%%%%%%%%%%%%%%%%%%%%%%%%%%%%%%%
\documentclass{svmult}

% Choose options for [] as required from the list in the Reference Guide

\usepackage{mathptmx}			% selects Times Roman as basic font
\usepackage{helvet}         		% selects Helvetica as sans-serif font
\usepackage{courier}        		% selects Courier as typewriter font
\usepackage{type1cm}        		% activate if the above 3 fonts are
                            				% not available on your system
						%
\usepackage{makeidx}      	   	% allows index generation
\usepackage{graphicx}        		% standard LaTeX graphics tool
                             				% when including figure files
\usepackage{multicol}        		% used for the two-column index
\usepackage[bottom]{footmisc}		% places footnotes at page bottom
\usepackage{amsmath,amssymb,epsfig}
\usepackage{epstopdf}

% See the list of further useful packages in the Reference Guide

\makeindex    % used for the subject index
                      % please use the style svind.ist with
                      % your makeindex program

%%%%%%%%%%%%%%%%%%%%%%%%%%%%%%%%%%%%%%%%%%%%%%%%%%%%%%%%%%%

\begin{document}
\title*{Gaze-based control of a swarms of robots using electroencephalography} 	% Fix!

% Use \titlerunning{Short Title} for an abbreviated version of your contribution title if the original one is too long

\author{Luca Mondada$^1$, Mohammad Ehsanul Karim$^2$, and Francesco Mondada$^2$}
\authorrunning{L Mondada, M E Karim and F Mondada}

% Use \authorrunning{Short Title} for an abbreviated version of your contribution title if the original one is too long

\institute{$^1$Department of Physics, Swiss Federal Institute of Technology ETHZ, Z\"urich, Switzerland, \email{Email address} \\
$^2$Laboratoire de Syst\`emes Robotiques, Ecole Polytechnique F\'ed\'erale de Lausanne, Lausanne, Switzerland, \email{Email address}} % Fix!


% Use the package "url.sty" to avoid
% problems with special characters
% used in your e-mail or web address

\maketitle

\abstract{
	Literature shows, several gesture and speech recognition based techniques have been explored to perform interactive control of swarm-robots. Indeed, search and rescue operations could benefit from an efficient control of a human over a group of robots. However such techniques are not quite intuitive in establishing connection between the user and a single robot. This study systematically investigates electroencephalography (EEG) signals to detect the robot the user is looking at and subsequently select it based on analyzing the steady-state visually evoked potential (SSVEP). SSVEP has a repeatable neural response to a regularly blinking visual stimulation that varies depending on the visual stimulation frequency. In our experiments, the LEDS on the robots blinked at different frequencies, and the corresponding SSVEP signal was analyzed to detect and select the robot.\\
	This study systematically studies several factors (EEG signal type, algorithm, distance, and LED color) which directly impacts the efficiency of the system. Based on the studies proposes a methodology and critically analyze it based on the performance of 10 subjects.
}

\keywords{Emotiv EPOC, SSVEP, EEG, swarm robotics, human-swarm interactions, human-robot interaction}

\section{Introduction}
\label{sec:introduction}
Distributed multi-robot systems have extremely promising applications; for example: search for rescue, environmental monitoring, or geographical mapping. Therefore the topic has been extensively studied under various names: swarm robotics~\cite{brambilla2013} or collective robotics~\cite{kernbach2013handbook}. To-date researches and engineers have struggled in designing scalable, robust, efficient (compared to single robot) and affordable distributed multi-robot systems. In addition to the autonomous control strategies, there are multiple challenges from the human-robot interaction (HRI) point of view. Although some single-robot control interfaces have shown promising results; however, efficiently controlling a robotic swarm is still a state-of-the-art problem.

One of the major HRI challenges in attempting to control several robots simultaneously is robot selection: \textit{How to robustly establish connection between a particular member of the swarm and the human-controller?} Re-phrased otherwise, how can we select one particular robot from the swarm? Fong et al. have proposed a simple selection protocol which uniquely identifies each robot using a numbering system; the selection and manipulation was performed through remote control-based communication \cite{fong2003}. However such systems become complex, confusing, and difficult to manage as the swarm grows. Various other methods: Gesture recognition \cite{Nagietal2014, Jones2010, Couture-Beil2010, Monajjemi2013}, robot-vision based user-gaze interpretation \cite{Couture-Beil2010, Monajjemi2013, Pourmehr2013} and speech recognition \cite{Pourmehr2013} have been used too. Goodrich and others provided a decent summary of the field \cite{yanco2004classifying, goodrich2007human, Rule2012}. Some of these ideas could work in theory; however none of these have been tested in practice. The major practical short-coming involves issues pertaining to ergonomy and intuitiveness. 

As a novel approach, this study analyzes electroencephalography (EEG) signals to detect the robot the user is looking at and subsequently select it based on analyzing the steady-state visually evoked potential (SSVEP). In particular, the study compares two reliable and well-documented EEG neural responses - the P300 and the SSVEP \cite{Zhu2010, Bi2013, Beverina2003}. P300 neural response is elicited by salient stimulations. SSVEP is measured when a visual stimulus is repeatedly shown at a certain frequency. Although the P300 response has been given more attention; recent studies show target selection can be efficiently achieved using SSVEP because computational analysis can reliably distinguish multiple SSVEP responses corresponding to multiple frequencies \cite{SSVEPfiability}. Therefore, the frequency of the blinking light can be used to detect the target being watched by the EEG user. For all the aforementioned reasons, the robot selection using EEG is studied in this work. For the EEG signal analysis several processing chains have been suggested, for details please refer to the survey in \cite{Bi2013}. In our case, we tested two algorithms: a machine learning approach \cite{openvibeSSVEP}, and a signal processing algorithm using a modified version of Lin's CCA separator \cite{Lin2014}.

	This study systematically studies several factors (algorithm, distance, LED color) which directly impacts the efficiency of the system. Based on the studies proposes a methodology and critically analyze it based on the performance of 10 subjects.

% Fix
The following section details general material and methods. Because of the specific use of SSVEP, some key parameters have been explored in preliminary experiments presented in section \ref{sec:opti}. A robot selection experiment with 10 subjects is then described in section~\ref{sec:expe}, followed by the discussion and conclusions.

\subsection{Paragraphs and indents}

As you will see from the result of this sample paper, the class file will produce subsection titles that are all boldfaced. The indents of the paragraphs are done automatically. Please note that the first line of text that follows a heading is not indented, whereas the first lines of all subsequent paragraphs are. Instead of simply listing headings of different levels we recommend to let every heading be followed by at least a short passage of text.

For unnumbered lists we recommend to use the \textit{itemize} environment -- it will automatically render Springer's preferred layout.

\section{Figures}

Figures should be centered and referenced in the text as, for example, Fig.~\ref{gear}. Please use adequate letter fonts (including size and type) and line width. The example follows. As you see in Fig.~\ref{gear}, the figure caption is automatically generated in the proper format.

\begin{figure}
\center
\includegraphics[width=0.35\textwidth]{figure}
\caption{Please write your figure caption here}\label{gear}
\end{figure}

\section{Equations}

Equations should be centered and numbered. They should be referenced in the text as, for instance, Eq. (\ref{Newton}). Please use adequate letter fonts (including size and type). An example of equation is as follows
\begin{equation}\label{Newton}
\sum\limits_{i = 1}^n {{\mathbf{F}}_{i} }  = m \mathbf{a}_G
\end{equation}


\section{Tables}

Tables should be centered and referenced in the text as, for
example, Table \ref{tri}.

\begin{table}
\caption{Please write your table caption here}
\label{tri}
\centering
\begin{tabular}{p{2cm}p{2.4cm}p{2cm}p{2.4cm}}
\svhline\noalign{\smallskip}
Description 1 & Description 2 & Description 3 & Description 4  \\
\hline\noalign{\smallskip}
Row 1, Col 1 & Row 1, Col 2  & Row 1, Col 3 & Row 1, Col 4\\
Row 2, Col 1 & Row 2, Col 2  & Row 2, Col 3 & Row 2, Col 4\\
Row 3, Col 1 & Row 3, Col 2  & Row 3, Col 3 & Row 3, Col 4\\
\svhline\noalign{\smallskip}
\end{tabular}
\end{table}






\section{Referencing}

The reference list should be in an alphabetical order. The provided
class file will produce the format you see, which is correct. A citation
in the paper should have the form \cite{contrib,journal,mono,online} for a contribution in conference proceedings, journal article, monograph and on line document respectively. You can also use bibliography style \textit{spmpsci} in BibTeX.


\section{Conclusions}
Your manuscript must be sent following the instructions and deadlines given on the MeSRob web page \url{www.mesrob2015.irccyn.ec-nantes.fr/}.

\begin{acknowledgement}
The MeSRob 2015 Local Committee wants to acknowledge the previous MeSRob organizers' help.
\end{acknowledgement}

\begin{thebibliography}{99.}
% Use the following syntax and markup for your references

% Contribution in conference proceedings
\bibitem{contrib} Ceccarelli, M.: The Challenges for Machine and Mechanism Design at the Beginning of the Third Millennium as Viewed from the Past. In:  Proceedings of Brazilian Congress on Mechanical Engineering COBEM2001, Uberlandia, Invited Lecture, pp. 132-151 (2001)
% Journal article
\bibitem{journal} Husty, M. L. and Gosselin C.: On the singularity surface of planar 3-RPR parallel mechanisms. Mech. Based Design of Structures and Machines, \textbf{36}, 411--425 (2008)
% Monograph
\bibitem{mono} Merlet, J.-P.: Parallel Robots (Series: Solid Mechanics and Its Applications). Springer (2006)
% Online Document
\bibitem{online} Viadero, F. \textit{et al.}: A model of spur gears supported by ball bearings. In: Computational Methods and Experimental Measurements XIII (CMEM) Invited lecture (2007) Available via WIT eLibrary. \\
\url{http://library.witpress.com/pages/PaperInfo.asp?PaperID=18033}

\end{thebibliography}

% Or you can use BibTeX
%\bibliographystyle{spmpsci}
%\bibliography{mybib}


\end{document}

















